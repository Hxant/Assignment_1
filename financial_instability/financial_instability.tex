\documentclass[12pt,a4paper]{article}
\usepackage[utf8]{inputenc}
\usepackage{amsmath}
\usepackage{amsfonts}
\usepackage{amssymb}
\author{Mokili Isaac Janda RegNo: 16/U/7096/PS StudNo: 216013042}
\title{A General Research Paper On Financial Instability At The Beginning Of The Year}
\parskip 2ex
\begin{document}
\maketitle
Over the past few years in Uganda, it is common for society to face a problem of financial instability at the the beginning of every new year, a dangerous observation that is slowly and steadily becoming a trend.

Financial instability is the state of having limited monetary resources to aid in accomplishing a number of essential requirements that need urgent attention at a specific period of time. Practical examples include; the need to provide basic needs including food, medicine, clothing in households, the need to pay school fees and tuition at respective learning institutions most especially at the start of the year, the need to meet unforeseen circumstances that require immediate intervention, among others.

The issue at hand arises due to avoidable reasons but with human nature being a major driving force in the decisions we make, it takes a toll on the reasoning capacity of many individuals in society and as such, people tend to live in the present and disregard the many dangers looming close by in the near future. With such a driving force, many individuals tend to proudly manifest the 'financial illiteracy' they possess and hence adopt poor spending habits towards the end of the year. 

With human nature being the meat of the issue at hand, the festive season becomes it's backbone and together they take the front seat in driving society to financial instability at the beginning of the year. 

To illustrate further, it is evident that the festive season has largely programmed society to become 'spending machines' in a notion of being merry and jubilant and hence towards achieving that goal, a lot of money is spent unnecessarily without proper budgeting thus only bringing about financial doom to society at the start of the year.

The festive season still is a period of idleness and individuals normally termed as 'workaholics' often can't stand to sit around and do nothing and in turn degenerate to 'alcoholics' as an excuse to push time. This degeneration leads only to reckless spending habits since one's reasoning is incapacitated by alcohol and hence multitudes of money go to waste towards not only buying oneself liquor but also impressing friends and 'friends of friends'.

In an effort to provide for their families during the festivities, people in higher authorities most especially the law that is, traffic officers and police tend to become unnecessarily harsh to society for minor offenses and as such, in order to escape jail time, an individual is easily swayed to pay out huge sums of money in form of bribes to these officers and as a result the individual later fails to strike a balance when fulfilling his/her own festive needs and essential expenses at the beginning of the year thus financial instability at that time.

From a criminal point of view, with all the merry and jubilation flying around during the festive season, society tends to become reckless and forgets all the dangers looming in the night and therefore invite trouble with open arms. This recklessness crops up crime which by no coincidence is rampant at the end of the year. These crimes range from robbery, theft, kidnapping, all of which involve large sums of money in form mercy for life, ransoms, medical bills, therefore, inevitably leading to financial instability at the start of the year.

With regard to the few observations above, the festive season, good as it may be should be celebrated with care as it can pose a threat to the welfare of society since money despite not being regarded a basic need is nevertheless essential in acquisition of most, if not all the basic needs required in every household.
\end{document}